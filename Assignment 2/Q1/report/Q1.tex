\documentclass{article}
\usepackage{graphicx}
\usepackage[top=0.5in,bottom=0.5in,  left=0.75in, textheight=8in, right=0.75in]{geometry}
\usepackage{amsmath}
\title{Q1}
\date{\today}
\author{
	Tejesh Raut\\
	\texttt{140050008}
	\and
	Deep Modh\\
	\texttt{140050002}
	\and
	Chaitanya Rajesh\\
	\texttt{140050073}
}
\begin{document}
	\maketitle
	Extend the outer boundaries to get the corners of outer boundary.
	\newline
	This new image formed in stored as wembley\_corners.jpg in output folder. Note that all the coordinates referred hereafter are with respect to wembley\_corners.jpg file and not the original wembley.jpeg.
	\newline
	\newline
	Find the homography matrix using the points to map the coordinates of outer boundary to an imaginary plane with coordinates to be the distance in yards:
	\newline
	(30, 371) - (0, 18)
	\newline
	(337, 271) - (0, 0)
	\newline
	(878, 323) - (44, 0)
	\newline
	(831, 553) - (44, 18)
	\newline
	Homography matrix hence obtained is:
	\newline
	\begin{math}
	\begin{pmatrix}
	0.000769943767210617&0.00236372736533829&-0.900041165556824 \\
	-0.000175535686156667&0.00182624627327294&-0.435757213821539 \\
	0.00000182250199699356&0.0000505176750096942&-0.00565706121197567
	\end{pmatrix}
	\end{math}
	\newline
	Use this homography matrix to find the mapping of corners of inner boundary.
	\newline
	a1 = (750, 505) bottom-right corner 
	\newline
	a2 = (821, 327) top-right corner
	\newline
	a3 = (357, 420) centre of bottom boundary
	\newline
	Length will be twice of distance between a1 and a3 and breadth will be distance between a1 and a2.
	\newline
	\newline
	Results obtained:
	\newline
	\textbf{Length: 36.75 yards}
	\newline
	\textbf{Breadth: 15.32 yards}
\end{document}